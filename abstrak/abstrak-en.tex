\begin{center}
  \large\textbf{ABSTRACT}
\end{center}

\addcontentsline{toc}{chapter}{ABSTRACT}

\vspace{2ex}

\begingroup
% Menghilangkan padding
\setlength{\tabcolsep}{0pt}

\noindent
\begin{tabularx}{\textwidth}{l >{\centering}m{3em} X}
  \emph{Name}     & : & \name{}         \\

  \emph{Title}    & : & \engtatitle{}   \\

  \emph{Advisors} & : & 1. \advisor{}   \\
                  &   & 2. \coadvisor{} \\
\end{tabularx}
\endgroup

% Ubah paragraf berikut dengan abstrak dari tugas akhir dalam Bahasa Inggris
\emph{The development of Human Computer Interaction or HCI technology which makes it possible to communicate with a computer using hand poses. With the advancement of technology, a remote control for robots begins to be developed by prioritizing user interaction with the robot. The control system on the robot has begun to leave the joystick remote control and switch to control using hand gestures or autonomously. The use of remote control joysticks is becoming obsolete because there are several deficiencies in controlling the robot. In this research, hand pose-based robot control will be made using a Convolutional Neural Network. In this study, the joystick remote control will be replaced with a camera as input for the user's hand pose. The image captured by the camera will later be classified using the Convolutional Neural Network to translate commands from the user to the robot. Classification results will later be sent to the robot using the internet and will be translated into navigation commands by the microcontroller.}

% Ubah kata-kata berikut dengan kata kunci dari tugas akhir dalam Bahasa Inggris
\vspace{2ex}
\noindent
\emph{Keywords}: \emph{Convolutional Neural Network}, \emph{Mobile Robot}, \emph{Pose}.
