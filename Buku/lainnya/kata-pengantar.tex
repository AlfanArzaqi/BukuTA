\begin{center}
  \Large
  \textbf{KATA PENGANTAR}
\end{center}

\addcontentsline{toc}{chapter}{KATA PENGANTAR}

\vspace{2ex}

% Ubah paragraf-paragraf berikut dengan isi dari kata pengantar

Puji syukur kehadirat Tuhan Yang Maha Esa, atas rahmat dan hidayah-Nya, sehingga penulis dapat menyelesaikan Tugas Akhir yang berjudul \textbf{“KENDALI MOBILE ROBOT BERBASIS POSE TANGAN MENGGUNAKAN CONVOLUTIONAL NEURAL NETWORK (CNN)”} dengan baik dan tepat waktu. Tugas akhir ini disusun sebagai salah satu persyaratan akademik untuk memperoleh gelar Strata 1 (S1) pada  Program Studi Teknik Komputer Institut Teknologi Sepuluh Nopember Surabaya. Pada kesempatan ini penulis ingin menyampaikan terima kasih kepada beberapa pihak yang telah membantu dalam penyusunan Tugas akhir diantaranya:

\begin{enumerate}[nolistsep]

  \item Allah SWT atas segala rahmat dan ridho-Nya sehingga penulis dapat diberi kelancaran dan kemudahan dalam mengerjakan Tugas Akhir

  \item Orang tua penulis serta kakak yang selalu memberikan doa dan dukungan baik secara moral maupun materi, sehingga penulis dapat menyelesaikan tugas akhir dengan lancar.
  
  \item Bapak Dr. Supeno Mardi Susiki Nugroho, S.T., M.T. selaku Kepala Departemen Teknik Komputer, Fakultas Teknologi Elektro dan Infomatika Cerdas, Institut Teknologi Sepuluh Nopember.

  \item Bapak Ahmad Zaini, S.T., M.Sc. selaku dosen pembimbing I serta Bapak Dr. Eko Mulyanto Yuniarno, S.T., M.T. selaku dosen pembimbing II yang selalu memberikan motivasi dan arahan dalam penyusunan Tugas Akhir.
  
  \item Bapak dan Ibu dosen pengajar serta staff departemen teknik komputer ITS atas pengajaran, bimbingan, serta perhatian yang diberikan kepada penulis selama ini. 
  
  \item Seluruh teman-teman dari angkatan e59 dan Teknik Komputer.

\end{enumerate}

Semoga Allah SWT membalas semua pihak yang telah membantu penulis dalam menyelesaikan tugas akhir. Akhir kata dengan segala kekurangan dan keterbatasan penulis dalam penyusunan tugas akhir ini, penulis berharap adanya saran dan kritik yang membangun, sehingga tugas akhir ini dapat bermanfaat bagi siapapun yang membacanya.

\begin{flushright}
  \begin{tabular}[b]{c}
    \place{}, \MONTH{} \the\year{} \\
    \\
    \\
    \\
    \\
    \name{}
  \end{tabular}
\end{flushright}
