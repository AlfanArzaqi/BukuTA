% Atur variabel berikut sesuai namanya

% nama
\newcommand{\name}{Alfan Miftah Arzaqi}
\newcommand{\authorname}{Musk, Elon Reeve}
\newcommand{\nickname}{Alfan}
\newcommand{\advisor}{Ahmad Zaini, S.T., M.Sc.}
\newcommand{\coadvisor}{Dr. Eko Mulyanto Yuniarno,S.T.,M.T.}
\newcommand{\examinerone}{-}
\newcommand{\examinertwo}{-}
\newcommand{\examinerthree}{-}
\newcommand{\headofdepartment}{Dr. Supeno Mardi Susiki Nugroho, ST., M.T.}

% identitas
\newcommand{\nrp}{0721 19 4000 0003}
\newcommand{\advisornip}{19750419200212 1 003}
\newcommand{\coadvisornip}{19680601199512 1 009}
\newcommand{\examineronenip}{-}
\newcommand{\examinertwonip}{-}
\newcommand{\examinerthreenip}{-}
\newcommand{\headofdepartmentnip}{19700313199512 1 001}

% judul
\newcommand{\tatitle}{KENDALI \textit{MOBILE ROBOT} BERBASIS POSE TANGAN MENGGUNAKAN \textit{CONVOLUTIONAL NEURAL NETWORK} (CNN)}
\newcommand{\engtatitle}{\emph{MOBILE ROBOT CONTROL BASED ON HAND POSE USING CONVOLUTIONAL NEURAL NETWORK (CNN)}}

% tempat
\newcommand{\place}{Surabaya}

% jurusan
\newcommand{\studyprogram}{Teknik Komputer}
\newcommand{\engstudyprogram}{Computer Engineering}

% fakultas
\newcommand{\faculty}{Teknologi Elektro dan Informatika Cerdas}
\newcommand{\engfaculty}{Intelligent Electrical and Informatics Technology}

% singkatan fakultas
\newcommand{\facultyshort}{FTEIC}
\newcommand{\engfacultyshort}{ELECTICS}

% departemen
\newcommand{\department}{Teknik Komputer}
\newcommand{\engdepartment}{Computer Engineering}

% kode mata kuliah
\newcommand{\coursecode}{EC224801}
