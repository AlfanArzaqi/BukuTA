\begin{center}
  \large\textbf{ABSTRACT}
\end{center}

\addcontentsline{toc}{chapter}{ABSTRACT}

\vspace{2ex}

\begingroup
% Menghilangkan padding
\setlength{\tabcolsep}{0pt}

\noindent
\begin{tabularx}{\textwidth}{l >{\centering}m{3em} X}
  \emph{Name}     & : & \name{}         \\

  \emph{Title}    & : & \engtatitle{}   \\

  \emph{Advisors} & : & 1. \advisor{}   \\
                  &   & 2. \coadvisor{} \\
\end{tabularx}
\endgroup

% Ubah paragraf berikut dengan abstrak dari tugas akhir dalam Bahasa Inggris
\emph{Initially the robot could only be controlled closely, but in the next few years the robot could be controlled remotely without wires or wirelessly and controlled with a joystick remote control. Robot control using hand gestures has begun to be developed using sensors placed on the hands and can also use computer vision to determine human body gestures. Among the computer vision algorithms, there is a method, namely the Convolutional Neural Network (CNN) which can efficiently process data in the form of images. The challenge of making hand gestures to be developed as robot control is that there are variations in the size of the human hand and there are disturbances when taking hand images using a camera. There is a method to recognize hand gestures using landmarks. Therefore, hand pose-based control of the mobile robot was created using a Convolutional Neural Network (CNN). The accuracy value obtained by testing 500 data is 100\% and the ideal distance is between 40cm and 120cm. The influence of position and angle produces ideal conditions when the hand is facing the camera. The robot has a stable response time from a distance of 100cm to 1000cm with 20 data received according to the data sent from the classification results.}

% Ubah kata-kata berikut dengan kata kunci dari tugas akhir dalam Bahasa Inggris
\vspace{2ex}
\noindent
\emph{Keywords}: \emph{Convolutional Neural Network}, \emph{Mobile Robot}, \emph{Pose}.
