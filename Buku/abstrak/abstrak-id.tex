\begin{center}
  \large\textbf{ABSTRAK}
\end{center}

\addcontentsline{toc}{chapter}{ABSTRAK}

\vspace{2ex}

\begingroup
% Menghilangkan padding
\setlength{\tabcolsep}{0pt}

\noindent
\begin{tabularx}{\textwidth}{l >{\centering}m{2em} X}
  Nama Mahasiswa    & : & \name{}         \\

  Judul Tugas Akhir & : & \tatitle{}      \\

  Pembimbing        & : & 1. \advisor{}   \\
                    &   & 2. \coadvisor{} \\
\end{tabularx}
\endgroup

% Ubah paragraf berikut dengan abstrak dari tugas akhir
Awalnya robot hanya dapat dikendalikan secara dekat, namun beberapa tahun berikutnya robot sudah bisa dikendalikan dengan jarak yang jauh dengan tanpa kabel atau \textit{wireles} dan dikendalikan dengan \textit{remote control joystick}. Kendali robot menggunakan gestur tangan mulai dikembangkan menggunakan sensor yang diletakkan pada tangan dan juga dapat menggunakan \textit{computer vision} untuk mengetahui gestur tubuh manusia. Di antara algoritma \textit{computer vision} ada suatu metode yaitu \textit{Convolutional Neural Network} (CNN) yang dapat memproses suatu data berupa citra dengan efisien. Tantangan pembuatan gestur tangan untuk dikembangkan sebagai kendali robot yaitu terdapat variasi terhadap ukuran tangan manusia serta terdapat gangguan saat pengambilan citra tangan menggunakan kamera. Terdapat metode untuk mengenali gestur tangan menggunakan landmark. Maka dari itu dibuat kendali mobile robot berbasis pose tangan menggunakan \textit{Convolutional Neural Network} (CNN). Nilai akurasi yang didapatkan dengan dari pengujian 500 data adalah 100\% dan didapatkan jarak ideal antara 40cm sampai 120cm. Pengaruh posisi dan sudut dihasilkan keadaan ideal disaat tangan menghadap kamera. Robot memiliki waktu respon yang stabil dari jarak 100cm sampai 1000cm dengan 20 data yang diterima sesuai data yang dikirimkan dari hasil klasifikasi.

% Ubah kata-kata berikut dengan kata kunci dari tugas akhir
\vspace{2ex}
\noindent
\textbf{Kata Kunci: \textit{Convolutional Neural Network}, \emph{Mobile Robot}, Pose,}
