\begin{center}
  \large\textbf{ABSTRAK}
\end{center}

\addcontentsline{toc}{chapter}{ABSTRAK}

\vspace{2ex}

\begingroup
% Menghilangkan padding
\setlength{\tabcolsep}{0pt}

\noindent
\begin{tabularx}{\textwidth}{l >{\centering}m{2em} X}
  Nama Mahasiswa    & : & \name{}         \\

  Judul Tugas Akhir & : & \tatitle{}      \\

  Pembimbing        & : & 1. \advisor{}   \\
                    &   & 2. \coadvisor{} \\
\end{tabularx}
\endgroup

% Ubah paragraf berikut dengan abstrak dari tugas akhir
Perkembangan teknologi \emph{Human Computer Interaction} atau HCI yang memungkinkan untuk berkomunikasi dengan komputer menggunakan pose tangan. Dengan majunya teknologi suatu \emph{remote control} pada robot mulai dikembangkan dengan mengutamakan interaksi pengguna dengan robot. Sistem kendali pada robot sudah mulai meninggalkan \emph{remote control} \emph{joystick} dan beralih pada kendali menggunakan gestur tangan atau secara \emph{autonomous}. Penggunaan \emph{remote control} \emph{joystick} mulai ditinggalkan karena terdapat beberapa kekurangan dalam mengendalikan robot. Dalam penelitian ini akan dibuat kendali robot berbasis pose tangan menggunakan \emph{Convolutional Neural Network}. Pada penelitian ini \emph{remote control} \emph{joystick} akan digantikan dengan kamera sebagai input pose tangan dari user. Citra yang ditangkap oleh kamera nantinya akan diklasifikasikan menggunakan \emph{Convolutional Neural Network} untuk menerjemahkan perintah dari pengguna kepada robot. Hasil klasifikasi nantinya akan dikirimkan kepada robot menggunakan internet dan akan diterjemahkan menjadi perintah navigasi oleh mikrokontroler. 

% Ubah kata-kata berikut dengan kata kunci dari tugas akhir
\vspace{2ex}
\noindent
\textbf{Kata Kunci: \textit{Convolutional Neural Network}, \emph{Mobile Robot}, Pose,}
