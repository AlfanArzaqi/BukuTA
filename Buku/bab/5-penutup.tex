\chapter{PENUTUP}
\label{chap:penutup}

% Ubah bagian-bagian berikut dengan isi dari penutup
Pada bab ini akan dipaparkan kesimpulan dari hasil pengujian yang akan menjawab dari permasalahan yang diangkat oleh pelaksanaan tugas akhir ini. Pada bab ini juga dipaparkan saran mengenai hal yang dapat dilakukan untuk mengembangkan penelitian kedepannya.

\section{Kesimpulan}
\label{sec:kesimpulan}

 Pada Tugas Akhir ini telah dijelaskan pengembangan Kendali \emph{Mobile Robot} Berbasis Pose Tangan Menggunakan \emph{Convolutional Neural Network} (CNN). Dari hasil metodologi yang telah dilakukan didapatkan nilai akurasi sebesar100\% pada uji coba dengan menggunakan 500 data dan pada pengujian yang telah dilakukan pada Bab \ref{bab4} dengan variasi pengujian jarak kamera terhadap tangan, pengaruh posisi dan sudut tangan terhadap kamera, data citra dari responden, dan jarak komunikasi. Pada pengujian jarak kamera didapatkan jarak ideal untuk dapat mengendalikan robot adalah 40cm sampai dengan 120cm dengan nilai akurasi 91\% sampai 100\%. Pada pengujian pengaruh posisi dan sudut tangan didapatkan bahwa posisi tengah mendapatkan kondisi ideal disaat tangan di serongkan ke kiri yang mendapatkan nilai akurasi sebesar 94\%, sedangkan pada posisi kanan, kiri, dan atas kamera didapatkan kondisi ideal disaat tangan dihadapkan ke arah kamera dengan nilai akurasi 100\%. Pada pengujian data citra dari responden didapatkan nilai akurasi sebesar 95\% pada Responden 1 yang merupakan laki-laki dewasa. Pada pengujian jarak komunikasi robot didapatkan jarak 100cm sampai 1000cm 20 data dapat uji terkirim dan sesuai dari kode instruksi yang dikirimkan dan mendapatkan rata-rata waktu respon sebesar 805ms.

\section{Saran}
\label{chap:saran}

Untuk kerperluan pengembangan dari Tugas Akhir ini saran yang dapat dipertimbangkan sebagai berikut :

\begin{enumerate}[nolistsep]

  \item Mencoba penggunaan gestur untuk kendali robot.
  \item Mencoba pada robot yang memiliki kemampuan yang berbeda. 


\end{enumerate}
